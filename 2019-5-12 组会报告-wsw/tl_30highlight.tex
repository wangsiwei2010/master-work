%%%%%%%%%%%%%%%%%%
\section{无人机竞赛}

\begin{frame}
    \frametitle{无人机目标检测竞赛}
    \begin{itemize}
        \item 在已有的无人机云台上,需要我们去建立算法来实现目标检测
        \item one-stage和two-stage的目标检测算法的迁移
        \item 空军的迫切需求
        \item 学术转到工程上的考虑:实际需求
    \end{itemize}
\end{frame}

\begin{frame}
    \frametitle{缺失k-means聚类}
    \begin{itemize}
        \item 首届空军主办的无人机挑战赛,在西安举办
    \end{itemize} 
    \begin{figure}
        \includegraphics[width=0.9\textwidth]{12.jpg} 
    \end{figure}          
\end{frame}

\begin{frame}
    \frametitle{无人机目标检测竞赛}
    \begin{itemize}
        \item 现有的识别方法基于传统的CV算法,如hough圆变换等等,但是会有很多不确定因素
        \item 比如目标是红色,但是不同的天气下红色在图像中的像素不一样,会导致错检甚至于无法识别
        \item 深度学习的方法是可以来处理这些变化的,基于纹理特征
        \item 此次采用的是retinanet来处理
    \end{itemize}
\end{frame}

\begin{frame}
    \frametitle{无人机目标检测竞赛}
    \begin{itemize}
        \item 首届空军主办的无人机挑战赛,在西安举办
    \end{itemize} 
    \begin{figure}
        \includegraphics[width=0.9\textwidth]{13.jpg} 
    \end{figure}          
\end{frame}

\begin{frame}
    \frametitle{无人机目标检测竞赛}
    \begin{itemize}
        \item 首届空军主办的无人机挑战赛,在西安举办
    \end{itemize} 
    \begin{figure}
        \includegraphics[width=0.4\textwidth]{14.jpg} 
    \end{figure}          
\end{frame}

\begin{frame}
    \frametitle{比赛不足}
    \begin{itemize}
        \item 训练不够充分,设备未准备好
        \item 图传性能不好,传回的图片质量较差
        \item 无人机的高度不固定的时候,目标的尺度不一,而此次比赛就缺少对于小样本的注意力
        \item 机载和工作站的区别
    \end{itemize}
\end{frame}